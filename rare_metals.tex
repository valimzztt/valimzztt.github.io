\documentclass[letterpaper,10pt]{article}
\usepackage[a4paper, total={6.5in, 9in}]{geometry}
\usepackage{graphicx}
\usepackage{comment}
\usepackage{chemformula}
\usepackage[
  backend=biber,
  style=numeric,
  sorting=none
]{biblatex}
\addbibresource{references.bib}

\begin{document}
\title{How rare earths metals power our digital future, and the risks we must consider}
\author{Author: Valentina Mazzotti\\Audience: Non-specialists with a general knowledge of science and physics\\Ideal publication outlet: Scientific American or The Conversation}
\date{}
\maketitle

There is something incredibly small inside your laptop, your smartphone, and even the MRI machine at your local hospital that makes them all work. It’s not just silicon or iron, but the rare earth elements. These metals, with bizarre names such as gadolinium (Gd), dysprosium (Dy), lanthanum (La), and lutetium (Lu), unknown to most people, are behind the operation of modern digital technologies and at the core of the renewable energy transition. From the magnets in wind turbines and electric vehicles to lanthanum-based alloys in nickel-metal hydride batteries, rare earth elements are essential to many technologies we rely on every day.
\newline  
\par
But what exactly are rare earth elements, and why are they considered so essential to the scientific and economic landscape of the 21st century? Rare earth elements (REEs) are a group of 17 chemically similar metals: the 15 lanthanides, plus scandium and yttrium \cite{Gupta1992}. Despite their name, most are relatively abundant in Earth’s crust \cite{Rowlatt2014}. For instance, cerium, the most abundant rare earth element, is more common than copper. The “rare” designation originates from the fact that they are rarely found in concentrated deposits and are typically spread out in small amounts, naturally mineralized together and requiring chemical separation for technological use.
\newline  
\par
Rare earth elements can be combined with other materials to form compounds that exhibit magnetic, luminescent, and catalytic properties that are extremely difficult to replicate \cite{Gschneidner1964}. Because of these properties, REEs have become critical components in high-performance materials used across a wide range of industries, enabling improvements in efficiency, durability, and miniaturization of electronic components. 

\includecomment{Most rare earth elements posses unpaired electrons in their 4f orbitals, allowing them to contribute large  magnetic dipole moments when embedded in specific crystal structures} %\cite{Boysen2011}
Most rare earth elements possess electronic properties that make them excellent candidates for strong, permanent magnets. For instance, neodymium (Nd), dysprosium (Dy), and praseodymium (Pr) are commonly used to create high-performance magnets \cite{Dent2012, aemree_dysprosium_2025, Voncken2016}, which are essential in wind turbines, electric vehicle motors, computer hard drives, and even in the tiny speakers of your earbuds \cite{stanfordmagnets_neodymium}.
\newline  
\par
Among them, neodymium is perhaps the best known due to its role in the strongest commercial magnets available today. While pure neodymium is only magnetic at very low temperatures \cite{wikipedia_neodymium}, it becomes technologically useful when combined with iron and boron to form the compound \ch{Nd_2Fe_{14}B}. This material exhibits the highest known magnetic energy density, with neodymium magnets storing up to 18 times more magnetic energy than iron magnets of the same volume. This remarkable performance illustrates a key point: it is not rare earth elements alone that give rise to desirable properties, but rather their combination in small, tailored amounts with other elements. When engineered into compounds like \ch{Nd_2Fe_{14}B}, rare earths unlock entirely new functionalities making them indispensable to modern technology. The applications of these “super magnets” are broad, spanning industries from automotive, where they enable smaller and lighter motors, to clean energy, such as in wind turbines, where they produce a magnetic field that does not need an external power source. Although rare earth materials make up only a small fraction of a device’s weight or volume, they are often essential to its function. For instance, rare earth-based magnets may constitute only a minor fraction of a laptop’s total mass, but they are critical for components such as the spindle of a disk drive, which is necessary for data storage.  Without them, many core components of modern electronics would be significantly bulkier, less efficient, or simply not work at all.
\newline  
\par
Beyond magnetism, other rare earth elements play crucial roles in optical and catalytic applications. Europium (Eu), terbium (Tb), and yttrium (Y) are used in phosphors that produce the vivid colors seen in LED displays and energy-efficient lighting \cite{Zhang2017}. Cerium (Ce) is essential in catalytic converters for reducing vehicle emissions and in high-efficiency glass polishing \cite{Dey2020, Janovs2016}. The list could go on, making it clear how these elements are ubiquitous in today’s technology: from the screens we look at, to the cars we drive, to the tools we use to make science.
\newline  
\par
Despite their critical role, rare earth elements are associated with several challenges, not least of which are environmental and geopolitical concerns. Mining and refining rare earths can produce radioactive waste and cause soil and water contamination if not properly managed. In some countries, weak environmental regulations in mining practices have led to deforestation, pollution, and human rights abuses. The global supply chain is also heavily concentrated: China accounts for over 70\% of global annual rare earth mine production, estimated at 210,000 tonnes in 2022 \cite{NaturalResourcesCanada2022}, raising concerns about supply security, trade dependence, and potential market manipulation.

As technology advances, rare earth elements will continue to play a critical role, especially in sustainable energy and high-tech devices. However, concerns about the availability and sustainable recovery of these materials have grown, leading to their classification as critical raw materials due to their supply risk \cite{european_commission_critical_2023}. These concerns call for innovative solutions that reduce dependence on primary mining, such as improved recycling techniques and the development of alternative materials. Another promising area is improving environmentally friendly separation techniques of rare-earth metals, such as chemical leaching \cite{Peelman2016} and advanced solvent extraction \cite{zhang2023}, which could make their processing less harmful.
\newline  
\par
Rare earth elements might not be the first materials we think of when considering what our economies rely on, but they are central to many of the most visible technological transformations of our time. When studying these materials with an eye toward their industrial applications, understanding the science and policy behind rare earth elements must be seen as more than an academic exercise. It is a critical consideration, because ensuring a resilient, responsible, and equitable supply of these elements will be essential to advancing a more sustainable future.
\printbibliography
\end{document}
